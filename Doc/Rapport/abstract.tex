\renewcommand{\abstractnamefont}{\normalfont\Large\bfseries}
%\renewcommand{\abstracttextfont}{\normalfont\Huge}

\begin{abstract}
\hskip7mm

\begin{spacing}{1.3}

L'objectif de ce projet, réalisé dans le cadre de l'UE PSAR (4I408)
est de concevoir un logiciel pédagogique pour l'apprentissage de la
logique du premier ordre. Ce logiciel apportera notamment un support
visuel sous la forme d'un jardin et de fleurs permettant d'appréhender
plus facilement les formules de la logique. Les étudiants auront à
leur disposition deux fenêtres : l'une permettant de concevoir un
jardin en y positionnant des fleurs et l'autre permettant d'écrire des
formules à l'aide d'un clavier visuel. Ils pourront alors soit
construire un jardin respectant un ensemble de formule données, ou au
contraire établir des formules à partir d'un jardin donné. Les
formules pourront être analysées de manière automatique afin de
vérifier leur cohérence. Enfin, les étudiants pourront sauvegarder
leurs jardins et leurs formules pour les réutiliser ultérieurement.
\newline
La première partie de ce rapport présente plus amplement le sujet, les outils mis à disposition pour le réaliser et les premières pistes pour résoudre les problématiques liées à leurs utilisation.
La seconde partie explique concrètement quels vont être les fonctionnalités à développer ainsi que l'organisation du travail dans le groupe et la structure général du code.
La partie suivante explore plus amplement les choix techniques qui ont été réalisé, leurs avantages et inconvénients ainsi que leurs utilisation et rôle dans la réalisation du projet.
Enfin, ce rapport se conclu sur une analyse des résultats obtenu en comparaison des attentes formulées dans le cahier des charges.

\end{spacing}
\end{abstract}
