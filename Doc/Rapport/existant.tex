\chapter{Analyse de l'existant}

Le projet tourne autour d'un script préalablement fourni en python.

\section{Script python}

Le script permet de créer un jardin en disposant les différents éléments avec leurs attributs sur diverses coordonnées.
En premier lieu il est nécessaire de bien créer un jardin, il serait inutile de tester une formule sans contexte autour.
Le script permet ensuite de créer une formule, celle-ci doit être préalablement vérifiée pour que sa syntaxe ne contienne aucun erreur.
Dans un dernier temps le script analyse la formule dans le contexte du jardin donné et retourne un booléen en résultat.


\section{Bilan récapitulatif}

Voici le tableau (cf. fig. 2.1) récapitulatif de l'analyse de l'existant:\\

%tableau centré à taille variable qui s'ajuste automatiquement suivant la longueur du contenu
\begin{figure}[!h]
\begin{center}
\begin{tabular}{|l|l|l|l|}
  \hline
  Solution & Création & Vérification & Analyse\\
  \hline
  Jardin & Oui & Non & Non \\
  Formule & Oui & Non & Oui \\
  \hline
\end{tabular}
\end{center}
\caption{Tableau récapitulatif des solutions}
\end{figure}
