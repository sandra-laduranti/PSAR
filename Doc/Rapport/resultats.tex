\chapter{Résultats}

Pour conclure il est de mise de faire un état des fonctionnalités finales du programme et de les comparer à ce qui était attendu initialement.

\section{Évolution des fonctionnalités}

Les objectifs concernant l'analyse syntaxique et la communication avec le script de base ont bien été atteints conformément aux pré-requis établis pour pouvoir avoir une interface graphique fonctionnelle. \\
Cependant nous avons du ajuster quelques technologies choisies lors de la réalisation du cahier des charges. Comme évoqué dans le chapitre 3.1.2 sur la Communication avec le script Python, nous avions à l'origine choisi de faire la liaison avec un script Ocaml qui s'est avéré être trop compliquée à mettre en place.
\\
Concernant l'interface graphique, de nombreux points avaient été mis en avant lors de la réalisation du cahier des charges et rappelés dans les chapitres précédents (voir chapitre 2.1 Besoins). Tous les objectifs principaux ont été atteints. Lors de l'avancée du projet, nous avions prévu de faire un système de \textit{Drag \& Drop} depuis un panel vers le jardin (voir chapitre 3.2.3 Le Jardin et les fleurs), nous avons du revoir nos exigences à la baisse devant le temps de programmation restant.
\\
Nous avons d'ailleurs, et ce tout du long de ce projet, tenu nos engagements par rapport au diagramme de Gantt fixé dans le cahier des charges (voir annexes). Bien que nous ayons espéré finir plus tôt afin de pouvoir ajouter des fonctionnalités supplémentaires.
\newline Dans les causes de petit retards, initialement pris en compte dans la réalisation du planning, on pourra noter la perte de temps avec le premier script Ocaml, les soucis avec GitHub qui bloque les commit de trop grande taille, le passage à un git sur un serveur privé qu'il a fallut monter nous même et donc impliquant des petits soucis de configuration à ajuster et enfin la configuration de Unity à ajuster pour pouvoir y faire passer des librairies externes.
\\
Toutes les fonctionnalités principales nécessaires pour le bon fonctionnement et la bonne utilisation du logiciel sont donc opérationnelles. Mais il est bien entendu toujours possible d'apporter de nouvelles améliorations aussi bien graphiques (améliorer et personnaliser l'aspect graphique de l'interface, création de nouveaux environnements, etc.) que fonctionnelles (ajouter un outil de "retour en arrière", permettre à l'utilisateur de définir des raccourcis pour l'utilisation du clavier, etc.).

